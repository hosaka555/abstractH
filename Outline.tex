%#BIBTEX jbibtex Outline
%\documentstyle[listings,jlisting,txfonts]{grad-abst}

\documentclass[uplatex, 10pt, a4p]{jsarticle}

\usepackage{grad-abst}
\usepackage{listings, jlisting}
\usepackage[dvipdfmx]{graphicx}

\usepackage{verbatim}           % コメントアウトしてくれる便利なプリアンブルが使える \begin{comment} ... \end{comment}
\usepackage{txfonts}
\usepackage{setspace}
\usepackage{url}

\usepackage[dvipdfmx]{hyperref}
\usepackage{pxjahyper} %%hyperref読み込みの直後に

%\addtolength\oddsidemargin{-2.2zw}  % 魔法の呪文01
%\addtolength\evensidemargin{-2.2zw} % 魔法の呪文02
%\addtolength\textwidth{4.4zw}       % 魔法の呪文03

\setcounter{page}{1}

\newcommand{\thesisToolName}{Prot}
\newcommand{\thesisExtensionToolName}{ProtExtension}


\title{OpenModelicaのシミュレーション結果を用いたモータ特性表自動生成ツールの試作}
\author{原田 海人}

\major{情報システム工学科}
\lab{片山(徹)研究室}

\setstretch{0.9}

%\setlength{\columnseprule}{0.3mm}\setlength{\columnseprule}{0.3mm}

\begin{document}
\maketitle


\section{はじめに}
近年ソフトウェア開発の効率化が求められており、設計段階や製品を試作する前に、製品の機能や性能を検証したいというニーズが高まっている\cite{modelicaモデルベース本}。
このニーズに応えるために、モデルベースシステム開発手法がある\cite{modelicaモデルベース本}。モデルベースシステム開発手法とは、製品の設計を元にシミュレーションツールを用いて、
シミュレーションを行いながら、設計品質の向上を図る開発手法である\cite{ipa_2016}。本手法は、組込みシステムの開発において特に重要である\cite{ipa_useful_modelbase_dev}。

モデルベースシステム開発手法を用いた組込みシステムの開発に、OpenModelica\cite{open_modelica}が使われる。
OpenModelicaは、Modelica\cite{modelicaモデルベース本}コードのモデリング、シミュレーション、デバッグのための機能などを持つオープンソースプラットフォームである。
OpenModelicaが出力するシミュレーションの結果は、グラフや数値であり、csvファイルに出力できる。しかし、この出力を用いて、性能を決定付ける特定の値を確認するためには、手間と時間がかかる。
そこで、本研究では、性能を決定付ける特定の値を確認するためにかかる時間の削減を目的として、OpenModelicaのシミュレーション結果を用いたモータ特性表自動生成ツールの試作を行う。
特性表とは、製品の性能をまとめた一覧表であり、複数の製品の性能比較をする際に利用される[参考文献]。特性表を用いることで、性能の特定の値を容易に確認できるため、本研究の生成対象とする。また、本研究では、シミュレーションの対象として、ブラシ付きDCモータ[モータ使う]を対象とする。


\section{モーター特性表自動生成ツールの実装}\label{cha:OverviewFunction}
本研究で試作したモータ特性表自動生成ツールの実装について説明する。

\section{適用例}\label{sec:Indication}



\section{考察}\label{sec:Evaluation}

\section{おわりに}
本論文では、性能を決定付ける特定の値を確認するためにかかる時間の削減を目的として、モータ特性表自動生成ツールを試作した。
なお、本研究では、シミュレーションの対象として、ブラシ付きDCモータを対象とする。

モータ特性表自動生成ツールは、csvファイル解析部、特性表の要素算出部、モータ特性表生成部の3つの処理部で構成している。
csvファイル解析部では、csvファイルを読み込み、モータ特性表を生成するために必要なデータを抽出する。特性表の要素算出部では、抽出したデータからモータ特性表の各要素を算出する。
モータ特性表生成部では、算出したデータを基に特性表と、4つの特性グラフを生成し、これらを1つのPDFファイルにまとめ、モータ特性表として出力する。

適用例として、「ブラシ付きDCモータのModelicaモデル」と「ブラシ付きDCモータのModelicaモデルをサブシステムとするモデル」のシミュレーション結果ファイルを適用した結果、
2つのモデルから正しくモータ特性表を生成することを確認した。

考察の評価において、モータ特性表自動生成ツールの有用性を示すことができた。具体的には、モータ特性表自動生成ツールを用いることにより、特定の値を確認するためにかかる時間を削減できるかどうかを検証した。
検証にはXとYの2種類のケースを用意し、被験者4名を2グループに分けて、モータ特性表自動生成ツールを用いる場合と、用いない場合で実験を行った。

この実験結果により、ケースXについてはモータ特性表自動生成ツールを使用した場合では、使用しなかった場合に比べて被験者の回答時間を91.8\%削減できた。
ケースYについては、モータ特性表自動生成ツールを使用した場合では、使用しなかった場合に比べて被験者の回答時間を92.1\%削減できた。
この結果により、モータの性能を決定づける特定の値を確認するためにかかる時間を削減できたと言える。また、双方のケースにおいて、モータ特性表自動生成ツールを用いた場合、モータ特性表自動生成ツールを用いない場合に比べ、問題の正答率が上昇した。
% 正答率が上がった理由として、モータ特性表自動生成ツールを用いる
% 人手によるミスを削減できたことが考えられる。具体的には、モータ特性表自動生成ツールを用いない場合、手動で表計算ソフトから特定の値を抽出し、計算を行う必要があるため、
% この過程においてミスが発生する可能性が高まったことが考えられる。
% 一方、モータ特性表自動生成ツールを用いた場合、ツールが自動で特性表を生成し、提示することにより、人手によるミスが発生する可能性を削減できたことが考えられる。

以下に、今後の課題を示す。

\begin{itemize}
    \item 対象とするモータのモデルが1種類しかない\\
    本論文で試作したモータ特性表自動生成ツールが対象とするのはブラシ付きDCモータである。しかし、ブラシレスモータやACモータなどには対応していない。
    そのため、それらを用いた回路のシミュレーション結果からモータ特性表を作成できない。

\item 特性表の要素をユーザが変更できない\\
      本論文で試作したモータ特性表自動生成ツールが生成する特性表は、12個の要素を持つ。
      しかし、ユーザがこの要素を変更することはできない。

\item 特性グラフを出力形式をユーザが指定できない\\
      本論文で試作したモータ特性表自動生成ツールは、4つの特性グラフを個別に出力する。
      しかし、特性グラフは1つのグラフに複数の要素を表示することで、グラフの比較を行いやすくなる。


\end{itemize}



\footnotesize
\bibliography{bibtex} %bibファイルの.bibの前の部分


\bibliographystyle{junsrt} %引用された順番に出力
\end{document}

